\section{Введение}

\epigraph{
    \todo{These are not the droids you are looking for}
}{
    \todo{Obi-Wan Kenobi}
}

Задача повышения производительности программ стоит перед разработчиками с самого начала эпохи компьютерных технологий.
Более эффективные алгоритмы позволяют провести больше экспериментов и расчетов за единицу времени,
сэкономить вычислительные ресурсы, повысить удовлетворенность пользователей программных решений.

Масштабы и сложность существующих программных систем постоянно растут, зачастую экспоненциально.
Размеры крупных вычислительных центров могут достигать нескольких сотен тысяч серверов.
Заметно увеличиваются расходы крупных компаний на серверные компоненты.
При эксплуатации программных систем подобных размеров любые оптимизации исполняемого кода влекут значительную экономию ресурсов, оправдывающую время, затраченное программистами на оптимизацию и улучшение программы.

\subsection{Актуальность}
Существует множество подходов к оптимизации программ, начиная с простого <<метода пристального взгляда>>
и заканчивая использованием сложных систем со встроенным ИИ.
Одним из наиболее эффективным на практике является анализ программы при помощи профилировщика (profiler)
--- специализированного инструмента, позволяющего оценить использование ресурсов программой
и выдать результаты анализа в понятном и читаемом для человека формате.
По результатам анализа узких мест при помощи профилировщика программисты точечно оптимизируют программу,
повышая эффективность использования одного из самых дорогих ресурсов в разработке --- времени инженера.

Одной из важных проблем, ограничивающих применимость вышеописанного метода, является сложность воспроизведения реальной нагрузки на программу в тестовом окружении.
Например, часто оказывается, что копия программы, запущенная программистом на локальной машине, заметно отличается по характеристикам производительности от той же программы, запущенной на вычислительном кластере или промышленных серверах.
Можно выделить несколько важных ограничений: 
\begin{itemize}
    \item Различия в вычислительных характеристиках серверов, выполняющих анализируемый код (производительность и особенности CPU, скорость доступа и пропускная способность RAM, характеристики дисковой и сетевой подсистем).
        Например, зачастую разработчик запускает и анализирует код на относительно слабом железе, например, ноутбуке, при этом обобщая результаты анализа на мощные сервера, количественные характеристики которых могут превосходить таковые у ноутбуков на порядки.
        Известно множество алгоритмов, которые плохо масштабируются с ростом вычислительной мощности. Например, если сложность алгоритма растёт хотя бы линейно от числа доступных процессорных ядер.
    \item Сложность воспроизведения реальной нагрузки. 
        Например, для тестирования производительности веб-сервера необходимо собрать большое количество реальных запросов и выполнить их заново с тем же распределением по времени.
        Для обеспечения безопасности и приватности пользовательских данных запросы необходимо обезличивать, тем самым ещё более отдаляя анализируемую ситуацию от реальной.
    \item Отличия в программном обеспечении между реальным окружением и тестовым стендом (версии и особенности OS, версии драйверов и микрокода процессора).
    \item Особенности архитектуры облака.
        В современных облаках используется большое количество систем контейнеризации и виртуализации, позволяющих выполнять программы от разных пользователей в полной изоляции друг от друга, квотировать использование ресурсов, надежно контролировать выполнение пользовательских процессов и управлять ими.
    \item Отличия между версиями программного обеспечения, анализируемого разработчиком и версиями программы в реальном окружении.
        Например, зачастую код, исполняемый под реальной нагрузкой, компилируется с использованием link time optimization \cite{thinlto}: крайне дорогостоящий процесс оптимизации кода во время компоновки.
        Использование LTO заметно меняет профиль исполнения кода, однако сборка с LTO занимает заметно больше времени (вплоть до десятков минут) по сравнению со сборкой без LTO.
        Программисту намного удобнее проводить короткие итерации экспериментов с использованием обычной оптимизированной сборки, однако усложняется интерпретация результатов.
    \item Реальное окружение само по себе редко является гомогенным.
        В современных промышленных облаках часто соседствуют серверы различной конструкции.
        Разработчик часто просто не сможет подобрать релевантное окружение.
\end{itemize}

Кроме того, процесс профилирования зачастую требует много времени разработчика: необходимо собрать тестовый стенд, удостовериться в его репрезентативности, провести тест приложения на производительность и записать результаты профилирования, после чего проанализировать их. Данный процесс часто занимает десятки минут, при этом легко пропустить какую-то из деталей и получить нерелевантный профиль.

В связи с вышеописанными проблемами все больше интереса вызывают методы для анализа производительности программ наживую, под реальной нагрузкой и без использования тестовых стендов.

\subsection{Цели и задачи}
В процессе дипломной работы решалась задача разработки и внедрения сервиса распределенного профилировщика
с рабочим названием Perforator в компании Яндекс.
Perforator должен стать единственным массовым решением для анализа производительности в компании,
автоматически подключиться к тысячам существующих сервисов,
позволить разработчикам без сложного процесса настройки автоматически получить точные профили использования важных ресурсов
(в первую очередь, тактов процессора).

Использование массового профилировщика с низким порогом входа заметно повысить утилизацию ресурсов, сократить время ответа сервисов пользователям, повысить прозрачность использования ресурсов внутри компании, собрать бесценную статистику стоимости и частоты использования различных библиотек.

В первую очередь, Perforator должен поддерживать анализ нативных програм
(компилирующихся в машинный код, предназначенный для исполнения процессором без промежуточных слоев трансляции).
В будущем возможна поддержка интерпретируемых и JIT-компилируемых языков.

Разработка ведется для операционной системы Linux, с учетом ее особенностей и возможностей.
В первую очередь реализуется поддержка платформы x86-64, в будущем возможна поддержка ARM64.
Код системы находится под NDA.
